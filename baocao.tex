\documentclass{article}
\usepackage[utf8]{inputenc}
\usepackage[T5]{fontenc} % Sử dụng tiếng Việt
\usepackage[fontsize=13pt]{scrextend}
\usepackage[paperheight=29.7cm,paperwidth=21cm,right=2cm,left=3cm,top=2cm,bottom=2.5cm]{geometry} % Căn chỉnh khổ A4
\usepackage{graphicx} % Thư viện chèn ảnh
\usepackage{float} % Đặt vị trí chèn ảnh
\usepackage{tikz} % Thư viện tạo khung bìa
\usetikzlibrary{calc}
\usepackage{indentfirst}%thư viện thụt đầu dòng
\renewcommand{\baselinestretch}{1.2}%giãn dòng 1.2
\setlength{\parskip}{6pt} 
\setlength{\parindent}{1cm} 
\usepackage{titlesec} %thư viện set up các kiểu chữ
\setcounter{secnumdepth}{4}%4heading
\titlespacing*{\section}{0pt}{0pt}{30pt} %heading 1
\titleformat*{\section}{\fontsize{16pt}{0pt}\selectfont \bfseries \centering}

\titlespacing*{\subsection}{0pt}{10pt}{0pt} %heading 2
\titleformat*{\subsection}{\fontsize{14pt}{0pt}\selectfont \bfseries}

\titlespacing*{\subsubsection}{0pt}{10pt}{0pt} %heading 3
\titleformat*{\subsubsection}{\fontsize{13pt}{0pt}\selectfont \bfseries \itshape}

\titlespacing*{\paragraph}{0pt}{0pt}{10pt} %heading 4
\titleformat*{\paragraph}{\fontsize{13pt}{0pt}\selectfont \itshape}
\usepackage{caption}

%setup hình
\renewcommand{\figurename}{\fontsize{12pt}{0pt}\selectfont \bfseries Hình}
\renewcommand{\thefigure}{\thesection.\arabic{figure}}
\captionsetup[figure]{labelsep=space}

%setup bảng
\renewcommand{\tablename}{\fontsize{12pt}{0pt}\selectfont \bfseries Bảng}
\renewcommand{\thetable}{\thesection.\arabic{table}}
\captionsetup[table]{labelsep=space}

\renewcommand{\theequation}{\thesection.\arabic{equation}} %đánh số phương trình
\renewcommand{\contentsname}{MỤC LỤC}% set lại tên mục lục
\renewcommand{\listtablename}{DANH MỤC BẢNG BIỂU}% set lại tên bảng biểu
\renewcommand{\listfigurename}{DANH MỤC HÌNH}% set lại tên danh mục hình
\renewcommand{\refname}{TÀI LIỆU THAM KHẢO}% set lại tên tài liệu tham khảo

\usepackage[unicode]{hyperref}


\newtheorem{theorem}{Định lý}[section]

%\usepackage{fancyhdr}
% Tạo đường kẻ dưới trang với độ dày 0.4pt
%\renewcommand{\footrulewidth}{0.4pt}
%\pagestyle{fancy}
%\lhead{dòng trên trái}
%\rhead{\includegraphics[width=1cm,height=1cm]{logohcmus.png}}
%\fancyfoot{}
%\lfoot{dòng dưới trái}
%\rfoot{dòng dưới phải}
%\fancyfoot[C]{\thepage} % Đặt số trang ở giữa dưới cùng

\begin{document}

\begin{titlepage}
%tạo cái viền
\begin{tikzpicture}[overlay,remember picture] 
    \draw [line width=3pt]
    ($(current page.north west) + (3.0cm,-2.0cm)$)
    rectangle
    ($(current page.south east) + (-2.0cm,2.5cm)$);
    \draw [line width=0.5pt]
    ($(current page.north west) + (3.1cm,-2.1cm)$)
    rectangle
    ($(current page.south east) + (-2.1cm,2.6cm)$);
\end{tikzpicture}

\begin{center}
   \vspace{-6pt} TRƯỜNG ĐẠI HỌC KHOA HỌC TỰ NHIÊN - ĐHQG TPHCM\\
   \textbf{\fontsize{16pt}{0pt}\selectfont KHOA ĐIỆN TỬ - VIỄN THÔNG}
\end{center}

%\vspace{0.25cm}
\begin{figure}[H]
    \centering
    \includegraphics[width=4.5cm]{logohcmus.png}
\end{figure}

\begin{center}
\fontsize{24pt}{0pt}\selectfont BÁO CÁO \\
\vspace{12pt}
\textbf{\fontsize{32pt}{0pt}\selectfont THỰC HÀNH ĐIỆN TỬ TƯƠNG TỰ VÀ SỐ   }
\end{center}
\vspace{1cm}
\hspace{6pt}\textbf{\fontsize{14pt}{0pt}\selectfont Bài:}
\begin{center}
    \textbf{\fontsize{20pt}{0pt}\selectfont GÕ CÁI ĐỀ TÀI VÔ}
\end{center}
\vspace{1cm}
\begin{table}[H]
    \centering
    \begin{tabular}{l l}
       \fontsize{14pt}{0pt}\selectfont Sinh viên thực hiện: &  \fontsize{14pt}{0pt}\selectfont NGUYỄN THÀNH ĐẠT - 23200074\\
         & \fontsize{14pt}{0pt}\selectfont HỒ QUANG ĐẠI - 23200069 \\
         & \fontsize{14pt}{0pt}\selectfont NGUYỄN PHƯỚC ĐẠT - 23200073\\
         & \fontsize{14pt}{0pt}\selectfont Lớp: 23DTV1-Nhóm --- \\
        \fontsize{14pt}{0pt}\selectfont Giảng viên hướng dẫn: & \fontsize{14pt}{0pt}\selectfont NGUYỄN THỊ THIÊN TRANG\\
        & \fontsize{14pt}{0pt}\selectfont PHAN VIỆT DŨNG\\
        & \fontsize{14pt}{0pt}\selectfont BÙI AN ĐÔNG
    \end{tabular}
\end{table}
\vspace{1cm}
\begin{center}
    \fontsize{14pt}{0pt}\selectfont TPHCM - 10/2024
\end{center}
\end{titlepage}

\cleardoublepage %xong 1 chương phải dùng dòng này để sang chương mới

%\thispagestyle{empty}
\section*{LỜI NÓI ĐẦU}
Phần này khái quát 1-2 trang về bối cảnh hình thành và mục đích đồ án, cá nhân góp phần trong thực hiện đồ án
%\cleardoublepage

%\thispagestyle{empty}
\section*{LỜI CAM ĐOAN}
Tôi tên là NGUYỄN THÀNH ĐẠT, mã số sinh viên 23ccdmdm, sinh viên lớp cc, khóa zz, người hướng dẫn TS.NGUYỄN THÀNH ĐẠT. Tôi xin cam đoan toàn bộ nội dung được trình bày trong đồ án ..... là kết quả quá trình tìm hiểu và nghiên cứu của tôi. Các dữ liệu nêu trong đồ án là hoàn toàn trung thực, phản ánh đúng kết quả thực tế. Mọi thông tin trích dẫn đều tuân thủ các quy định về sở hữu trí tuệ; các tài liệu tham khảo được liệt kê rõ ràng. Tôi xin chịu hoàn toàn trách nhiệm với những nội dung được viết trong đồ án này.

\vspace{6pt}
\hspace{7cm}TPHCM, ngày cc tháng dm năm 202cc

\hspace{9cm}\textbf{Người cam đoan}

\vspace{2cm}
\hspace{8.65cm}\textbf{NGUYỄN THÀNH ĐẠT}
%\cleardoublepage

\addtocontents{toc}{\protect\thispagestyle{empty}}
\tableofcontents %tạo mục lục tự động
\thispagestyle{empty}
\cleardoublepage

%\pagenumbering{roman}
%\section*{DANH MỤC KÝ HIỆU VÀ CHỮ VIẾT TẮT}
%\phantomsection\addcontentsline{toc}{section}{\numberline {} DANH MỤC KÝ HIỆU VÀ CHỮ VIẾT TẮT}
%\cleardoublepage

{\let\oldnumberline\numberline
\renewcommand{\numberline}{\figurename~\oldnumberline}
\listoffigures} % Tạo danh mục hình vẽ tự động
\phantomsection\addcontentsline{toc}{section}{\numberline{} DANH MỤC HÌNH VẼ}
\cleardoublepage

{\let\oldnumberline\numberline
\renewcommand{\numberline}{\tablename~\oldnumberline}
\listoftables}% Tạo danh mục bảng tự động
\phantomsection\addcontentsline{toc}{section}{\numberline{} DANH MỤC BẢNG BIỂU}
\cleardoublepage

\section*{TÓM TẮT ĐỒ ÁN}
\phantomsection\addcontentsline{toc}{section}{\numberline {}TÓM TẮT ĐỒ ÁN}
phần này trình bày mục đích và các kết luận quan trọng nhất của đồ án
\cleardoublepage

\pagenumbering{arabic}%đánh số thứ tự 1, 2, 3
\section*{CHƯƠNG 1. CHƯƠNG MỞ ĐẦU}
\phantomsection\addcontentsline{toc}{section}{\numberline{}CHƯƠNG 1. CHƯƠNG MỞ ĐẦU}
phần mở đầu giới thiệu vấn đề mà đồ án cần giải quyết, mô tả được các phương pháp hiện có để giải quyết vấn đề, trình bày mục đích đồ án song song với việc giới hạn phạm vi vấn đề mà đồ án giải quyết. Phần này giới thiệu tóm tắt cấu trúc đồ án và nội dung tương ứng các phần sẽ được trình bày ở các chương tiếp theo

Nội dung chính của đồ án tốt nghiệp bao gồm:
\begin{itemize}
    \item phần mở đầu giới thiệu đề tài.
    \item một chương giới thiệu cơ sở lí thuyết.
    \item một hoặc nhiều chương trình bày các vấn đề về tính toán và thiết kế.
    \item một chương mô tả các thí nghiệm và kết quả thu được.
\end{itemize}

\newpage
\section*{CHƯƠNG 2. CƠ SỞ LÝ THUYẾT}
\phantomsection\addcontentsline{toc}{section}{\numberline{}CHƯƠNG 2. CƠ SỞ LÝ THUYẾT}
\setcounter{section}{2}
Mỗi chương sẽ bắt đầu bằng một đoạn giới thiệu các phần chính được trình bày trong chương đó, dài khoảng từ 5 đến 10 dòng và kết thúc bằng một đoạn tóm tắt các kết luận chính của chương. Chú ý phân bố chiều dài chương cho cân đối hợp lí.
Mỗi chương sẽ bắt đầu bằng đoạn giới thiệu các phần chính được trình bày trong chương đó, dài khoảng từ 5 đến 10 dòng và kết thúc bằng một đoạn tóm tắt các kết luận chính của chương. Chú ý phân bố chiều dài chương cho cân đối hợp lý.\cite{nani2021antioxidant}
\subsection{Một số lưu ý khi trình bày đồ án}
sau đây là một vàu lưu ý khi làm đồ án các thằng cc nên nhớ nhé:
\subsubsection{Nộp đồ án}
nộp 2 bản cmm đi
\begin{itemize}
    \item được \textbf{in hai mặt} nhằm tiết kiệm không gian
    \item được đóng bìa mềm, bên ngoài là bóng kính
    \item số trang từ 50-150 trang, không kể phụ lục
    \item phải có chữ ký sinh viên sau LỜI CAM ĐOAN và của giảng viên hướng dẫn
\end{itemize}
\subsubsection{Phụ lục}
Phụ lục (nếu có) chứa các thông tin có liên quan đồ án nhưng để trong phần chính sẽ gây rườm rà. thông thường chi tiết được để trong phần phụ lục là: kết quả thô (chưa qua xử lý), mã nguồn phần mềm, thông số kĩ thuật chi tiết của linh kiện, hình ảnh minh họa thêm,...

\subsubsection{Tài liệu tham khảo}
\paragraph{Cách liệt kê} \mbox{}

áp dụng cách liệt kê theo quy định của IEEE. Theo đó, tài liệu tham khảo được đanh số trong ngoặc vuông. Thứ tự liệt ê là thứ tự xuất hiện của tài liệu tham khảo được trích dẫn trong đồ án. Tài liệu tham khảo bắt buộc phải được trích dẫn trong phần nội dung đồ án. Tài liệu tham khảo phải có nguồn gốc rõ ràng và phải từ nguồn đáng tin cậy, hạn chế trích dẫn tài liệu tham khảo từ các website, từ wikipedia.
\paragraph{Các loại tài liệu tham khảo}\mbox{}
các nguồn tài liệu chính là sách, bài báo trong các tạp chí, bài báo trong các hội nghị khoa học và các tài liệu tham khảo khác trên internet.
\subsubsection{Đánh số phương trình}
Phương trình được đánh số theo số của chương như hình vẽ và bảng biểu.
\subsubsection{đánh số định nghĩa, định lý, hệ quả}
các định nghĩa, định lý, hệ quả sẽ được đánh số theo số của chương sử dụng chung một chỉ số. ví dụ Định lý 3.1 định nghĩa 3.2, hệ quả 3.3


\newpage
\section*{CHƯƠNG 3. THUẬT TOÁN}
\phantomsection\addcontentsline{toc}{section}{\numberline{}CHƯƠNG 3. THUẬT TOÁN}
\setcounter{section}{3}
giới thiệu cc j giới thiệu đi
\subsection{Cách chèn ảnh }
\setcounter{figure}{0}%set lại số
\begin{figure}[H]
    \centering
    \includegraphics[width=0.5\linewidth]{logohcmus.png}
    \caption{\bfseries\fontsize{12pt}{0pt}\selectfontảnh này là cái logo trường tao}
    \label{hinh3.1}
\end{figure}
Hình \ref{hinh3.1} là ví dụ về cách chèn ảnh lưu ý chú thích của hình vẽ được đặt ngay dưới hình vẽ tất cả hình vẽ phải được đề cập trong phần nội dung phân tích và bình luận nhé các thằng ông nậu
\subsection{tạo bảng}
\begin{table}[H]
    \centering
     \caption{kết qua}
    \begin{tabular}{|c|c|c|c|}
    \hline
      1  &  1& 1 &1 \\\hline
        &  &  &\\\hline
        &  &  &\\\hline
      
    \end{tabular}
   
    \label{vailonluon}
\end{table}
bảng \ref{vailonluon} là ví dụ về cách tạo bảng, Tất cả các bảng biểu phải được đề cập trong phần nội dung và phải được phân tích, bình luận giống như bố m đang làm nhé

\subsection{Cách viết phương trình}
\begin{equation} \label{pt31}
    F(x) = \int^a_b \frac{1}{3}x^3
\end{equation}
phương trình

\subsection{Cách viết định nghĩa, định lỹ, hệ quả, bộ đề,...}
\begin{theorem}
    Một cc là một cc là một cc vcvcvc
\end{theorem}
\cleardoublepage
\section*{KẾT LUẬN}
\phantomsection\addcontentsline{toc}{section}{\numberline{}KẾT LUẬN}
trình bày mục đích và kết luận của ddood án
\subsection{Kết luận chung}
\phantomsection\addcontentsline{toc}{subsection}{\numberline{}Kết luận chung}
kết luận chung cho các chương trình trong đồ án. Mục này cần nhấn mạnh những vấn đề đã giải quyết và vấn đề chưa giải quyết để đưa ra mức độ hoàn thành công việc. đánh giá này thường bao gồm việc so sánh kết quả thu được so với mục tiêu ban đầu
\subsection*{Hướng phát triển}
\phantomsection\addcontentsline{toc}{section}{\numberline{}Hướng phát triển}
có nè
\subsection*{Kiến nghị và đề xuất}
\phantomsection\addcontentsline{toc}{section}{\numberline{}Kiến nghị và đề xuất}
nếu có
\cleardoublepage
\phantomsection\addcontentsline{toc}{section}{\numberline{}TÀI LIỆU THAM KHẢO}
\bibliographystyle{IEEEtran}
\bibliography{tailieuthamkhao}
trích dẫn dùng \cite{nani2021antioxidant}

\cleardoublepage
%\section*{Phụ lục}
\phantomsection\addcontentsline{toc}{section}{\numberline{}Phụ Lục}
\texttt{\fontsize{10pt}{0pt}\selectfont Mã nguồn chương trình (nếu có) được đưa vào đây, sử dụng font courier New, cỡ 10pt }
%\pagestyle{fancy} % kiểu trang như trên gất chi là fancy
\end{document}
